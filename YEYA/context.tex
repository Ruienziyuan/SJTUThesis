\chapter{选题背景}
\label{chap:context}

\section{中国制造2025战略}

根据《中国制造2025》计划,该计划以创新驱动、质量为先、绿色发展、结构优化、人才为本作为基本方针;以提高国家制造业创新能力、推进信息化与工业化深度融合、强化工业基础能力、加强质量品牌建设、全面推行绿色制造、推动重点领域突破发展、深入先进制造业结构调整、积极发展服务型制造和生产性服务业、提高制造业国际化发展水平作为任务和重点,并提供八大支撑和保障。该计划所提及的重点发展领域包括新一代信息技术创新产业、高档数控机床和机器人、航空航天装备、海洋工程装备及高技术船舶、轨道交通装备、节能与新能源汽车、核能或可再生能源电力装备、农机信息整合系统、纳米高新材料或模块化建筑及生物化学医药及高性能医疗器械\cite{zhizao2025}。

2016年8月18日,中国国务院工业和信息化部宣布首个“中国制造2025”试点城市为宁波。这意味着“中国制造2025”从指导文件开始转入具体实施阶段\cite{hualuoningbo}。

尽管如此,但是我们也要认清这个战略的一些事实。

未来中国制造业发展所面临的深层次困境和挑战,不仅要求中国政府和企业充分吸收和学习发达工业国家工业化过程中的普遍制度、政策安排和共性创新实践,更要求中国从自身独特的产业基础、人力资源、市场需求和文化特征出发,构建并不断提升其独特的制造业核心能力。比较研究和历史分析显示,任何工业强国都具有不易模仿、不易扩散的核心技术能力,而能够促成后发国家跃升成为工业强国的制度安排,必然要与该国制造业的核心技术能力相匹配。与核心技术能力相适应的制度安排,既具有发达工业国家制度安排的一般性,更具有路径依赖和一国独特能力所定的异质性,而只有制度安排中的那些异质性成分才能构成工业强国的组织能力,并与技术能力一起在具有“战略互补性”特征的演化过程中相互增强。但由于没有认识到中国制造业核心能力的特异性,目前国内学术界主流的制度观研究常常在复杂的实证分析之后提出一些各国共性的制度安排作为其对中国建设工业强国的政策建议,在这种情况下,其作为规范研究的意义自然就会大打折扣。与美、日、德、韩等工业强国相比,中国制造业的优势主要体现在模块化架构产品和大型复杂装备领域,而在产品架构一体化领域、制造工艺一体化领域以及既具有一体化特征又需要前沿科技支撑的核心零部件领域相对缺乏优势。未来中国制造业核心能力提升的可能方向一是通过架构创新和标准创新加强将一体化架构产品转化为模块化架构的能力缩短或者破坏产品生命周期演进的一般路径;二是针对国外技术与中国本土市场需求不匹配的机会,充分利用中国的市场和制造优势,不断提升复杂装备的架构创新和集成能力。以这样的学术理解为评价标准,《中国制造2025》本质上仅仅是一个政策力度更大的传统产业政策,而没有从根本上回答中国制造业“往何处去”和“如何去”的问题\cite{jianping2025}。

\section{主要矛盾的转化}

习近平在十九大报告中强调,中国特色社会主义进入新时代,我国社会主要矛盾已经转化为人民日益增长的美好生活需要和不平衡不充分的发展之间的矛盾。

自改革开放特别是 2001 年加入 WTO 以来, 中国积极融入全球经济体系, 充分利用国内外两个市场和两种资源,不断发挥自身的比较优势,在经济领域取得了举世瞩目的成就。然而,国际金融危机爆发以后, 世界经济持续低迷, 欧债危机愈演愈烈, 各国贸易保护主义重新抬头,在外部需求有所放缓和国内劳动力成本不断上涨的情况下,中国制造业正面临严峻挑战。在全球制造业调整的过程中,如何发现、培育以及发挥中国制造业的国际竞争力,向全球产业价值链的中高端发展, 对实现产业结构优化升级和提高世界竞争地位具有重要意义\cite{xianzhuang}。

\section{智能化趋势}

“智能”是当下比较火热的一个词,各种“智能”的产品、“智能”的制造如同雨后春笋般冒了出来。无论是“中国制造2025”还是德国“工业4.0”和美国“制造业振兴计划”,智能化都占据一个重要地位。

就工程机械方面来说。

我国工程机械智能化的发展起点较晚,且应用范围上还十分狭隘,在多个领域中都还尚未得到有效应用。面对当前我国建设规模不断扩大,工程机械走向智能化方向发展已成为必然的发展之路。这不仅是施工建设所需,也是社会进步所需。从我国当前工程机械智能化的应用趋势来看,我国当前应用较为广泛的为单机集成化、网络机群的智能化管理、工程机械智能化监控、维护、检测等技术,极大提高了施工项目的施工效率及质量。发展对策上,我国需实现综合技术的应用,开发工程机械故障诊断技术,将智能化工程机械应用人工智能、网络通信等,加大智能技术的应用领域及范围,加快工程机械的发展速度\cite{zhinenghua}。

显然不光是工程机械,在全制造业以及其他产业,智能化都已经成为一种趋势。
